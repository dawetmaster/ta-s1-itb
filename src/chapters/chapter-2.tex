\chapter{Studi Literatur}

% ---------------------------------------------------%
% DISCLAIMER:
% MENGINGAT BAB 2 ISINYA BEDA-BEDA, DI BAGIAN BAWAH ADA CONTOH DARI KATING. SUBBAB PAKAI \section, kalau mau 
%di sub-bin pake \subsection
% ---------------------------------------------------%
% EXAMPLE:
% \section{Pembelajaran dan Pemrograman}
% \subsection{Pembelajaran}
% Menurut \textcite{slavin2017learn}, belajar adalah perubahan yang relatif permanen dalam perilaku atau potensi perilaku sebagai hasil dari pengalaman atau latihan yang diperkuat. Belajar merupakan akibat adanya interaksi antara stimulus dan respons. Seseorang dianggap telah belajar sesuatu jika dia dapat menunjukkan perubahan perilakunya. Menurut teori ini, dalam belajar yang penting adalah input yang berupa stimulus dan output yang berupa respons.
